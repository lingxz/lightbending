%!TEX root = ../main.tex
\section{Introduction}\label{section:intro}

It has been established that the Universe is accelerating in its expansion, based on various complementary observations \citep{Spergel2003,Riess2004}. In the $\Lambda \text{CDM}$ cosmological model, this acceleration is powered by a cosmological constant $\Lambda$, that dominates the Universe's current energy budget. 

An active dispute that has been the subject of several papers in the last decade is whether the cosmological constant alters the bending of light. Given the undisputed success of general relativity, for which the deflection of light is one of the three classical tests \citep{Will1993}, and the popularity of the $\Lambda \text{CDM}$ model, one would think that the effect of a cosmological constant on gravitational lensing is well known. However, this is not the case. 

While there is general agreement that the influence of the cosmological constant on the bending of light, if any, is small (\citet{Rindler2007} in Eq. 17 estimates the influence of $\Lambda$ at roughly $10^{-28}$ of the neighbouring mass term), the possibility of doing precision cosmology with weak lensing as a tool renders the proposed difference significant enough. Importantly, if the cosmological constant is found to influence the deflection of light, gravitational lensing observations could allow for a new and independent constraint on $\Lambda$.
