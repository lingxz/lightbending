%!TEX root = ../main.tex
\section{Conclusion}\label{section:conclusion}

While most agree that the cosmological constant drops out of the null geodesic equations of the light ray (with some exceptions, for example \citet{Bhadra2010} and \citet{HAMMAD2013}), the debate was sparked again by \citet{Rindler2007} who argue that the cosmological constant shows itself in other ways in the metric. The discussion ensued, with many references agreeing with their conclusion while several others questioned it. Some references also suggest that the disagreement may be due to different interpretations of the $\Lambda$ terms in the lensing equation. 

The majority of work done in investigating the effect of a cosmological constant on the bending of light has been analytical rather than numerical. There have been attempts at settling the debate through numerical simulations \citep{beyon2012}, but they were neither complete nor conclusive enough. Indeed, current studies have already suggested directions in which numerical simulations can proceeed. 

While of no immediate practical significance, answering the question of whether the cosmological constant afffects gravitational lensing will be important for the next-generation precision cosmological. Given the importance of the cosmological constant in our current model of the universe, and as cosmological measurements become more precise, whether gravitational lensing observations can constrain the cosmological constant is a debate that deserves to be settled. 
