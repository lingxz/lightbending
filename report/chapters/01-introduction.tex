%!TEX root = ../thesis.tex
\chapter{Introduction}

There has been a debate over the past decade about whether the cosmological constant enters directly into the gravitational lensing equation. 

Till now, there is still no consensus as to whether $\Lambda$ contributes to lensing. 

\section{Previous work}

We are concerned about whether $\Lambda$ directly contributes to the bending of light around a concentrated mass. Conventional view, firt put forth by \citet{islam1983cosmological}, is that it does not, and classical lensing is correct as it is. 

Note that 


% The conventional view \cite{islam1983cosmological} states that the cosmological constant does not directly play a role in gravitational lensing, \cite{simpson2010lensing} apart from 

\section{Structure of this report}
In the remaining portions of this chapter I give an introduction of General Relativity and the basics of light propagation, which lays the mathematical foundation for this work. In \autoref{chapter:gravitational-lensing-formalism}, I provide an overview of the current established literature on gravitational lensing in a Schwarzschild spacetime where $\Lambda = 0$ and derivation of the key equations, before moving on to the case of a non zero $\Lambda$. 

The bulk of the work is in \autoref{chapter:swiss-cheese}, where I give the mathematical derivation of the equations which form the basis of this project, some of which do not appear explicitly in literature. In particular, in this chapter I describe the construction and mathematical properties of the Swiss-Cheese model with a Kottler condensation, and based on that, obtain the equations for light propagation in such a universe. 

Finally, in \autoref{chapter:results}, I present my numerical results for light propagation in such a universe and discuss their significance in the context of some of the analytical analyses that have been previously done. 

% notation
\section{A note on units and notation}
I use a comma to denote partial derivative and an overdot to denote derivative with respect to the affine parameter $\lambda$. For example, $x_{,t}$ refers to $\frac{\partial x}{\partial t}$ and $\dot{x} = \frac{dx}{d\lambda}$. Throughout this work I use natural units such that $c = G = 1$. 

\section{Background}

The essence of General Relativity (GR) is very elegantly summarized by John Wheeler into two parts: matter tells spacetime how to curve, and spacetime tells matter how to move \citep[pg.235]{wheeler2000geons}. 

The first half of this statement is quantified by the Einstein Field Equations, which describe the interaction between 

\begin{equation}
  G_{\mu \nu} + \Lambda g_{\mu \nu} = 8
  \label{eq:efes}
\end{equation}

In Genera Relativity (GR), Einstein's Field Equations (EFEs) describe the relation between matter and the geometry of spacetime. 

Einstein field Equa


In General Relativity, spacetime is described by a metric tensor $g$. 


This is the analogue of Poisson's equation in Newtonian gravity. 

The motion of a partice is described by a trajectory $x^{\mu}(\lambda)$

Objects move on a geodesic, which is a generalisation of the notion of "straight lines" to a curved spacetime. The equations can be derived 

The geodesic equation is 

\begin{equation}
  \ddot{x}^{\mu} + \Gamma^{\mu}_{\alpha \beta} \dot{x}^{\alpha} \dot{x}^{\beta} = 0 
  \label{eq:geodesic-eqn}
\end{equation}

where an overdot represents a derivative with respect to the affine parameter $\lambda$, and $\Gamma$ are the Christoffel symbols given by

\begin{equation}
  \Gamma^{\mu}_{\alpha \beta} = \frac{1}{2} g^{\mu \rho} (g_{\rho \alpha, \beta} + g_{\rho \beta, \alpha} - g_{\alpha \beta, \rho}).
  \label{eq:christoffels}
\end{equation}

\begin{equation}
  g_{\mu \nu} dx^{\mu} dx^{\nu} = 0
  \label{eq:null-condition}
\end{equation}

% EL equations