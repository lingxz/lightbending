%!TEX root = ../thesis.tex
\chapter{Introduction}

There has been a debate over the past decade about whether the cosmological constant enters directly into the gravitational lensing equation. 

\section{Previous work}
% The conventional view \cite{islam1983cosmological} states that the cosmological constant does not directly play a role in gravitational lensing, \cite{simpson2010lensing} apart from 

% notation
\section{A note on units and notation}
I use a comma to denote partial derivative and an overdot to denote derivative with respect to the affine parameter $\lambda$. For example, $x_{,t}$ refers to $\frac{\partial x}{\partial t}$ and $\dot{x} = \frac{dx}{d\lambda}$. Throughout this work I use natural units such that $c = G = 1$. 

\section{Background}

In General Relativity, spacetime is described by a metric tensor $g$. 

The Einstein field equations (EFE) 

This is the analogue of Poisson's equation in Newtonian gravity. 

The motion of a partice is described by a trajectory $x^{\mu}(\lambda)$

Objects move on a geodesic, which is a generalisation of the notion of "straight lines" to a curved spacetime. The equations can be derived 

The geodesic equation is 

\begin{equation}
  \ddot{x}^{\mu} + \Gamma^{\mu}_{\alpha \beta} \dot{x}^{\alpha} \dot{x}^{\beta} = 0 
  \label{eq:geodesic-eqn}
\end{equation}

where an overdot represents a derivative with respect to the affine parameter $\lambda$, and $\Gamma$ are the Christoffel symbols given by

\begin{equation}
  \Gamma^{\mu}_{\alpha \beta} = \frac{1}{2} g^{\mu \rho} (g_{\rho \alpha, \beta} + g_{\rho \beta, \alpha} - g_{\alpha \beta, \rho}).
  \label{eq:christoffels}
\end{equation}

\begin{equation}
  g_{\mu \nu} dx^{\mu} dx^{\nu}
  \label{eq:null-condition}
\end{equation}

% EL equations