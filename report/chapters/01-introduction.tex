%!TEX root = ../thesis.tex
\chapter{Introduction}

% There has been a debate over the past decade about whether the cosmological constant enters directly into the gravitational lensing equation. 

\section{Motivation and background}

It has been established that the Universe is accelerating in its expansion, based on various complementary observations \citep{riess2004type,spergel2003first}. In the $\Lambda \text{CDM}$ cosmological model, this acceleration is powered by a cosmological constant $\Lambda$, that dominates the Universe's current energy budget. Ironically, the idea of a cosmological constant was first pioneered by Einstein, who, being a believer of a static universe, introduced the term to keep his equations static, but later dropped it in favour evidence of an expanding universe, famously describing it as his biggest blunder. As fate would have it, the cosmological constant has made its way back into modern cosmology to account for the Universe's accelerating expansion. 

An active dispute that has been the subject of several papers in the last decade is whether the cosmological constant enters into the gravitational lensing equation. It is well known that light traveling through space is bent according to the mass distribution it encounters. This effect is gravitationa lensing, and it is one of the three classical tests of General Relativity \cite{will1993theory}. Since its first discovery in the 1970s, gravitational lensing has become an important observation tool in cosmology and astrophysics. Given the undisputed success of General Relativity, and the central role that $\Lambda$ now plays in gravitational physics and observational cosmology \citep{peebles2003cosmological}, one would think that the effect of a cosmological constant on gravitational lensing is well known. However, this is not the case. Scientific opinion has not converged on this issue and till now, there is still no consensus as to whether $\Lambda$ contributes directly to lensing. 
% One of the most important predictions is that 

While there is general agreement that the influence of the cosmological constant on the bending of light, if any, is small (\citet{rindler2007contribution} in Eq. 17 estimates the influence of $\Lambda$ at roughly $10^{-28}$ of the neighbouring mass term), the possibility of doing precision cosmology with weak lensing as a tool renders the proposed difference significant enough. Importantly, if the cosmological constant is found to influence the deflection of light, gravitational lensing observations could allow for a new and independent constraint on $\Lambda$.

\section{Previous work}

We are concerned about whether $\Lambda$ directly contributes to the bending of light around a concentrated mass. It is important to note that classical lensing already takes into account an implicit dependence on $\Lambda$ through the use of angular diameter distances, which will be explained in detail in \autoref{chapter:gravitational-lensing-formalism}. This is a $\Lambda$ effect that is known and already taken care of; therefore, when debating about whether $\Lambda$ contirbutes directly to lensing, we are really asking whether any modification to the current lensing formalism involving $\Lambda$ is needed. 

Conventional view, firt put forth by \citet{islam1983cosmological}, is that it does not, and classical lensing is correct as it is. This view, supported subsequently by multiple authors \citep{lake2002bending,park2008rigorous,simpson2010lensing,khriplovich2008does}, argues that the equations describing the path followed by a photon, the null geodesic equations, take the same form with or without $\Lambda$. Therefore, up till about a decade ago official opinion was that no modification to the current lensing theory involving a $\Lambda$ term is needed. 

% falls out of the exact equations of motion for the trajectory of the light ray. 
The main challenge to the conventional view came from \citet{rindler2007contribution}, who argue that while the $\Lambda$ term drops out of the equations of motion, $\Lambda$ still affects light bending through the metric of spacetime itself, since the photon is moving in $\Lambda$-dependent geometry. The angles that are measurable are defined by the metric, and since spacetime that includes a cosmological constant is not asymptotically flat, the process of measurement causes the cosmological constant to creep into the light bending angle. 

Since then, a plethora of papers have been written about this topic, but none have conclusively settled the debate. While there have supporting arguments in favour of Rindler and Ishak's proposal \citep{sereno2008influence,bhadra2010gravitational,schucker2008strong}, more recently there have been several arguments against it that question whether the influence of the cosmological constant has already been taken into account in the angular diameter distances and impact parameter in the formula \citep{butcher2016no,piattella2016lensing,arakida2012effect}. Indeed, most of the disagreements are about how the mathematical results should translate into observable quantities, especially since the pioneering analyses use a static metric that do not take into account the relative movement of the observer and source. The conflict comes from the fact that coordinate angles are not necessarily physically measurable, and \citet{lebedev2013influence} explore the question of measurable angles in detail. 

To date, most of the work done on this topic has been analytical. This inevitably lend some of the work to criticisms of whether the approximations used in deriving the results are valid (see for example criticisms in \citet{ishak2010more}). There have been some numerical work on this subject, but they have been few and far from comprehensive. 

For example, \citet{beynon2012testing} took a numerical approach in a Lema{\^\i}tre-Tolman-Bondi metric \citep{lemaitre1997expanding,tolman1934effect,bondi1947spherically} but did not reach a definitive conclusion. More recently \citep{aghili2017effect} used the McVittie metric \citep{mcvittie1933mass} in analyzing the effect of $\Lambda$. 

The most similar work was done in \citet{schucker2009strong}, who adopted a partially numerical approach in the model we are using (the Swiss-Cheese model) and concluded he agrees with Rindler and Ishak, but he only uses a single numerical example to reach a conclusion. Furthermore, some higher-order terms were dropped out in the integration in the Kottler metric and the contributiong of $\Lambda$ to the deflection was not singled out. He compares the results of \emph{different} models (pure Kottler versus Swiss-Cheese), both involving a non-zero $\Lambda$, to observed quantities, and hence his result are more inclined towards answering the question of which is the right physical model for gravitational lensing, as compared to the question we hope to answer, which is: What is the influence of the cosmological constant given a \emph{single} model, which we assume for the purpose of the analysis, and not without basis, is an accurate model of our universe? On this note we take a slightly different approach from Sch{\"u}cker, where we work with a single model (the Swiss-Cheese), and compare the results from a $\Lambda = 0$ universe with a universe in which we have a non-zero $\Lambda$. 

% state their shortcoming?
Our work will be numerical, and we hope to tackle some of the shortcomings of the previous numerical work, without falling into the approximation traps that exist in analytical work. In addition, we use a Swiss-Cheese model that deals with the problem of comoving observers in a cosmological setting and we use observable quantities throughout to compare the effect of $\Lambda$. The details of the model will be explained further in \autoref{chapter:swiss-cheese}.

\section{Structure of this report}
In the remaining portions of this chapter I give an introduction of General Relativity and the basics of light propagation, which lays the mathematical foundation for this work. In \autoref{chapter:gravitational-lensing-formalism}, I provide an overview of the current established literature on gravitational lensing in a Schwarzschild spacetime where $\Lambda = 0$ and derivation of the key equations, before moving on to the case of a non zero $\Lambda$. 

The bulk of the work is in \autoref{chapter:swiss-cheese}, where I give the mathematical derivation of the equations which form the basis of this project, some of which do not appear explicitly in literature. In particular, in this chapter I describe the construction and mathematical properties of the Swiss-Cheese model with a Kottler condensation, and based on that, obtain the equations for light propagation in such a universe. 

Finally, in \autoref{chapter:results}, I present my numerical results for light propagation in such a universe and discuss their significance in the context of some of the previous analytical analyses. 

% notation
\section{A note on units and notation}
I use a comma to denote partial derivative and an overdot to denote derivative with respect to the affine parameter $\lambda$. For example, $x_{,t}$ refers to $\frac{\partial x}{\partial t}$ and $\dot{x} = \frac{dx}{d\lambda}$. Throughout this work I use natural units such that $c = G = 1$. 
