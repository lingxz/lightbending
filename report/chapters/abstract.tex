%!TEX root = ../thesis.tex
% \begin{center}
%   \LARGE Abstract
% \end{center}
% %
% \noindent
% %
% \addchaptertocentry{Abstract}
\pdfbookmark[chapter]{Abstract}{abstract}
\chapter*{\centering Abstract}
% \addcontentsline{toc}{chapter}{Abstract}

There is an ongoing debate in the literature about whether the cosmological constant $\Lambda$ directly affects gravitational lensing. In this work we place this problem in a Swiss-cheese model with a Kottler vacuole, and numerically integrate null geodesics in this model to investigate the influence of $\Lambda$ on observable quantities in lensing. Our numerical results are compared with the predictions from conventional lensing analysis, \citet{rindler2007contribution}, and \citet{kantowski2010gravitational}, and we find that the numerical results agree best with Kantowski's predictions. However, it is difficult to isolate the influence of $\Lambda$ since it would necessarily change either the density or curvature of the FRW universe, both of which affect the propagation of light in a Swiss-Cheese universe. Nevertheless, our results estimate the effect of $\Lambda$ to be of the order of $10^{-6}$ or less, which is smaller than the second order mass terms commonly neglected in gravitational lensing analysis. 