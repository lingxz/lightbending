%!TEX root = ../thesis.tex
\chapter{Results and discussion}
\label{chapter:results}

We were able to reproduce the results presents in \citet{schucker2009strong}. 

A graph of results when we keep the lensing mass $M$ constant and vary $\Omega_{\Lambda}$ can be seen in [fig??]. On the $y$-axis, we have plotted the the deviation of $\alpha$ as a fraction of the standard FRW lensing case (given by \autoref{eq:series-expansion-R}), in order to put them on the same scale. [State the mass and zlens parameters] 

By varying the integration step size, we are able to estimate numerical errors on the integration. For a certain step size, we group the results obtained from the vicinity of step sizes together and find the variance in bending angles in that range. Fig [??] shows how the $\alpha$ obtained varies with step size. As is expected, the precision increases as we reduce the step size. 

Our results seem to follow the trend of Kantowski's predictions most closely, with a gap that reduces towards higher $\Lambda$. A possible explanation of this gap can be found by examining the neglected higher order term in Kantowski's predicted bending angle [eq??]. When $\Lambda = 0$, the ratio of this term to the leading order $(4M/r_0^2) \cos^3 \tilde{\phi_1}$ term is of the same order of magnitude as the fractional deviation of our numerical results from Kantowski's predictions. As is expected, this ratio decreases as $\Lambda$ when mass is kept constant, as can be seen from [fig??]. 

From the graph, we can see that even for the $\Lambda = 0$ case there is an offset between the numerical Swiss-Cheese result and the FRW prediction. Qualitatively, this is due to the fact that conventional lensing analyis assumes a mass superimposed on the homogeneous background, and this mass has infinite range. However, in the Swiss-Cheese model, the influence of the mass is limited, and bending stops once it leaves the Kottler hole. This is the main effect that Kantowski quantified in his paper \citet{kantowski2010gravitational}. This then begs the question of which model is a more accurate description of our physical universe, but this is not our primary concern. We are primarily concerned about whether $\Lambda$ has an influence on this effect. 

At this point I wish to note that the applicability of this work of course hinges on the validity of the model we use, but this can be said of most scientific analysis. The Swiss-Cheese model has been used in other areas in cosmology, not without success, as described in the introductory section of the previous chapter.  

There are a few different factors at play here. In discussing the results of this numerical integration, let us take a step back to look at the specific parts of ray-tracing that have a $\Lambda$-dependence. These are:
\begin{enumerate}
  \item The size of the hole. This is governed by \autoref{eq:junction-conditions-mass-volume}. In flat space, increasing $\Omega_{\Lambda}$ implies decreasing $\Omega_{\Lambda}$, which corresponds to the matter density of the universe. If we are to keep the mass constant, the hole size would have to increase as we increase $\Omega_{\Lambda}$. 
  \item The rate of expansion of the hole in static Kottler coordinates, given by \autoref{eq:hole-expansion-in-kottler-dR-dT}.  
  \item The Jacobian at the boundary, given by \autoref{eq:kottler-to-frw-transform-jacobian}.
\end{enumerate}

The first effect does not seem to be a truly direct $\Lambda$ effect, merely a side effect that in a flat universe, changing $\Omega_{\Lambda}$ must imply a change in matter density, but ultimately, it is the size of the hole that is the true determining factor. 

It is possible to keep the hole size constant while varying $\Lambda$ in flat space, but the mass would have to vary as well. This is shown in [ref??]. 

If we exten the integration for a universe with arbitrary curvature, then it is possible to fix both mass and $r_h$, but it would involve changing the curvature to compensate for the change in $\Lambda$. However, we can see in the previous chapter, the curvature $k$ affects the rate of expansion of the Kottler hole and also enters into the Jacobian at the boundary, so intuitively one would expect $k$ to have an effect on the bending angle in a Swiss-Cheese model as well, though this effect has not been explored in literature (Kantowski's calculation \citet{kantowski2010gravitational} only applies for flat space). 

Unfortunately, there is no way to keep both curvature and matter density constant while varying $\Lambda$, so we cannot truly isolate the effect of $\Lambda$. Ultmately, to compensate for a change in $\Lambda$, one has to either vary either matter density or curvature, or both. In a Swiss-Cheese model, both factors are expected to affect the lensing observables to some extent. Given the limited literature on the effect of curvature on lensing in the Swiss-Cheese model, and the much larger deviations resulting from curvature based on our numerical simulations [??], I would postulate that the variation of matter density has a smaller and better studied effect on lensing, and it is this we should vary such that the influence of $\Lambda$ becomes most apparent. Moreover, current cosmological observations \citep{ade2016planck,hinshaw2013nine,de2000flat} suggest quite convincingly that the Universe is spatially flat, and therefore looking at the case of $\Omega_k = 0$ will shed more light on our question. 

Hence, Figure [??] is the graph we should focus on. 

% compare with second order terms, check with third order terms, Simpsons & Heavens say it's routinely neglected.

% lensed vs unlensed values