%!TEX root = ../thesis.tex
\chapter{Conclusion}
\label{chapter:conclusion}

In this work, we examined gravitational lensing in a Swiss-cheese model that has an embedded Kottler hole. We numerically integrated null geodesics in such a universe piecewise and calculated lensing observables from the numerical results. These results were compared with predictions by Rindler \& Ishak and Kantowski. Our results agree most with Kantowski's predictions, with a small varying gap that can be explained by the presence of higher order terms not considered in Kantowski's results. 

However, it is difficult to truly isolate the effect of $\Lambda$, since a change in $\Lambda$ \emph{must} involve a change in either $\Omega_m$ or $\Omega_k$, and it is not clear at the outset which quantity should be kept constant as we turn up $\Lambda$. We looked into case where spatial curvature compensates for the change in $\Lambda$, but from our results it appears that $\Omega_k$ has a much larger effect than $\Lambda$ on lensing observables in the Swiss-cheese model, so we focused on interpreting results from a spatially flat Universe. 

We kept the hole size constant while varying $\Lambda$ to eliminate any effect coming from the size of the hole that might be considered a $\Lambda$ effect. Even so, $\Lambda$ appears to have an effect on the lensing observables, although this effect is small---smaller than the size of the second order term in $M/R_u$, which is routinely neglected. 

\section{Future work}

Our work is focused solely on the collinear case, in which the source, lens and observer are aligned, but it is straightforward to generalise beyond this restriction. 

The case with curvature was somewhat glossed over, since curvature appeared to have a much larger influence than $\Lambda$, and most work done on embedded lens in a Swiss-cheese have been in a spatially flat universe. While it is not the focus of this report, it might be worthwhile to look into the corrections from curvature needed in a Swiss-cheese model.  With better knowledge on the effect of curvature on lensing in such a model, it would be easier to isolate the $\Lambda$ effect in gravitational lensing and facilitate more effective comparison between the curved results and the spatially flat case. 

Lastly, it might be useful to extend this sytematic analysis beyond the Swiss-cheese model, possibly with the adoption of a different metric; after all, in our physical universe galaxies are not exactly completely spherical structures walled-off at a radius that varies strictly with its mass. \autoref{appendix:ltb} considers a more realistic (but still static) mass distribution, but it still operates in the Swiss-cheese framework. However, I believe this work to be a good starting point in quantifying the effect of the cosmological constant on lensing observables. 

% Other properties of gravitational lensing, such as shear and convergence, time delay.  
