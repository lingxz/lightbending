%!TEX root = main.tex
\documentclass[a4paper,12pt]{report}

% The packages are useful for many mathematical symbols but you may well not need them
\usepackage{amsmath,amssymb,amscd}
\usepackage{paralist}
\usepackage{bookmark}

\usepackage{caption}
% \usepackage{subcaption}
\usepackage[justification=RaggedRight,
            format=hang]{subcaption}

\usepackage{hyperref}
\usepackage[usenames,dvipsnames]{xcolor}
\hypersetup{
  colorlinks = true,
  citecolor  = RoyalBlue,
  linkcolor  = RoyalBlue,
  urlcolor   = RoyalBlue,
}

% \let\subsectionautorefname\sectionautorefname
% \let\subsubsectionautorefname\sectionautorefname

% This is great for reading in images
\usepackage{graphicx}
\usepackage{footnote}
\captionsetup[subfigure]{width=0.9\textwidth}

\usepackage[utf8]{inputenc}
\usepackage[backend=bibtex,style=authoryear,maxcitenames=2,natbib=true]{biblatex} % Use the bibtex backend with the authoryear citation style (which resembles APA)


\addbibresource{library2.bib}

% This is great for drafts as it will show all the labels you have defined.
% DON'T forget to comment it out before submission
%\usepackage{showkeys}

% Change Page Size
\typeout{--- Increasing width and height of text }
% A4 paper is 297mm high and 210mm wide.
\setlength{\textwidth}{16.00cm} % OK for both Letter and A4
\setlength{\oddsidemargin}{-0.04cm}  % actual margins = 1inch + \oddsidemargin
                                 %top/odd/even-sidemargin
\setlength{\evensidemargin}{-0.04cm} %  ditto
\setlength{\topmargin}{-1.0cm}      %  ditto
\setlength{\headheight}{18pt} \setlength{\headsep}{6pt}
\setlength{\topskip}{0pt}  %see pp155 also about baselineskip
\setlength{\textheight}{23.0cm} % 25cm for A4, 23cm for Letter or DJ
\setlength{\footskip}{0.7cm}



\renewcommand{\floatpagefraction}{.8}%

\usepackage[english]{babel}
\usepackage{chngcntr}
\counterwithout{figure}{chapter}
\addto\extrasenglish{%
  \renewcommand{\chapterautorefname}{Chapter}%
  \renewcommand{\figureautorefname}{Fig.}%
  \renewcommand{\equationautorefname}{Eq.}%
  \renewcommand{\sectionautorefname}{Section}%
  \renewcommand{\subsectionautorefname}{Section}%
}

\renewbibmacro{in:}{}

\usepackage[binary-units=true]{siunitx}
\DeclareSIUnit\px{px}
\DeclareSIUnit\parsec{pc}

\sisetup{%
  detect-all           = true,
  detect-family        = true,
  detect-mode          = true,
  detect-shape         = true,
  detect-weight        = true,
  detect-inline-weight = math,
}

\begin{document}

\begin{titlepage}
  \center

  \vspace*{2cm}
  \textsc{\Large Imperial College London}\\[0.5cm] 
  \textsc{\large Department of Physics}\\[0.5cm] 

  \title{\bf{Light bending in general relativity with a cosmological constant: a literature review}}
  \date{4 October, 2017}
  \author{Lingyi Hu\\ CID: \texttt{00919977}}

  {\let\newpage\relax\maketitle}
  \thispagestyle{empty}

  \vspace*{1cm}
  \noindent
  \large
  {\bf Project code:} ASTR-Heavens-1\par
  {\bf Supervisor:} Prof Alan \textsc{Heavens}\par
  {\bf Assessor:} Prof Carlo \textsc{Contaldi}

  \vspace*{1cm}

  {\bf Word count:}
  2478 words (excluding title page and bibliography)
  % 91 + 187 + 837 + 865 + 400 + 198

  \vspace*{2cm}


\end{titlepage}

\tableofcontents

\vspace*{2cm}
\noindent
% {\bf Abstract:} In this review, I survey some of the work analyzing the effect of the cosmological constant on gravitational deflection of light, a debate started mainly in a paper by \citet{Rindler2007} which challenged the conventional opinion. Since then, numerous papers have been written both in support and rebuttal of Rindler and Ishak. I go through the main arguments for and against a $\Lambda$-dependence in the lensing equation, identify limitations of some approaches, and look at the direction in which further work can take to settle this debate. 

% \newpage


%!TEX root = ../thesis.tex
\chapter{Introduction}

% There has been a debate over the past decade about whether the cosmological constant enters directly into the gravitational lensing equation. 

\section{Motivation and background}

It has been established that the Universe is accelerating in its expansion, based on various complementary observations \citep{riess2004type,spergel2003first}. In the $\Lambda \text{CDM}$ cosmological model, this acceleration is powered by a cosmological constant $\Lambda$ that dominates the Universe's current energy budget. Ironically, the idea of a cosmological constant was first pioneered by Einstein, who introduced the term to keep his equations static, but later dropped it when evidence showed otherwise. As fate would have it, the cosmological constant has made its way back into modern cosmology to account for the accelerating expansion of the Universe. Empirically, there is an onslaught of past cosmological data \citet{carmeli2001value,de2000flat,peebles2003cosmological} that our universe has a small but positive cosmological constant. 

An active dispute that has been the subject of several papers in the last decade is whether and consequently, how the cosmological constant affects the deflection of light. It is well known that light traveling through space is bent according to the mass distribution it encounters in an effect known as gravitational lensing. This phenomenon forms one of the important observational cornerstones of General Relativity \citep{will1993theory}. Since its first discovery in the 1970s, gravitational lensing has grown into one of the most deeply investigated phenomenon of gravitation and is becoming an increasingly important tool for observational astrophysics and cosmology. 

Given the undisputed success of General Relativity, and the central role that $\Lambda$ now plays in gravitational physics, one would think that the effect of a cosmological constant on gravitational lensing is well understood. However, this is not the case. Scientific opinion is divided on this issue and till now, there is still no consensus as to whether $\Lambda$ contributes directly to lensing. Debate persists on whether the conventional gravitational lensing formalism is already sufficient. While there is general agreement that the influence of the cosmological constant on the bending of light, if any, is relatively small, cosmological measurements are becoming more precise and the possibility of next-generation applications of lensing as a tool for cosmology only render the hanging debate more pressing. If the influence of the cosmological constant on light deflection is found to be different from the current prevalent expectation, future lensing observations will have to take this difference into account, and observations could very well allow for a new and independent constraint on $\Lambda$. 

Therefore, though perhaps not of immediate practical significance, the question of whether or how $\Lambda$ contributes to gravitational lensing is a debate that deserves to be settled. With this work we hope to shed some light on this dispute. 

% One of the most important predictions is that 

% While there is general agreement that the influence of the cosmological constant on the bending of light, if any, is small (\citet{rindler2007contribution} in Eq. 17 estimates the influence of $\Lambda$ at roughly $10^{-28}$ of the neighbouring mass term), the possibility of doing precision cosmology with weak lensing as a tool renders the proposed difference significant enough. Importantly, if the cosmological constant is found to influence the deflection of light, gravitational lensing observations could allow for a new and independent constraint on $\Lambda$.

\section{Previous work}

We are concerned about whether $\Lambda$ directly contributes to the bending of light around a concentrated mass. It is important to note that classical lensing already takes into account an implicit dependence on $\Lambda$ through the use of angular diameter distances, which will be explained in detail in \autoref{chapter:gravitational-lensing-formalism}. This is a $\Lambda$ effect that is known and already taken care of; therefore, when debating about whether $\Lambda$ contirbutes directly to lensing, we are really asking whether any modification to the current lensing formalism involving $\Lambda$ is needed. 

There has been ample literature on the topic. Conventional view, first put forth by \citet{islam1983cosmological}, is that no such modification is necessary, and classical lensing is correct as it is. This view, supported later by multiple authors \citep{lake2002bending,park2008rigorous,simpson2010lensing,khriplovich2008does}, argues that the equations describing the path followed by a photon, the null geodesic equations, are independent of $\Lambda$ and therefore the photon trajectory is wholly unaffected by the presence of a cosmological constant. This has been the official opinion up until about a decade ago. 

% falls out of the exact equations of motion for the trajectory of the light ray. 
The main challenge to the conventional view came from \citet{rindler2007contribution}, who argue that while $\Lambda$ drops out of the equations of motion, it still affects light bending through the metric of spacetime itself, since the photon is moving in $\Lambda$-dependent geometry. Physical measurable angles are defined by the metric, and since the spacetime metric itself includes a comological constant, the process of measurement causes the cosmological constant to creep into the light bending angle. 

Since then, a plethora of papers have been written about this topic, but none have conclusively settled the debate. While there have supporting arguments in favour of Rindler and Ishak's proposal \citep{sereno2008influence,schucker2008strong,bhadra2010gravitational}, more recently there have been several arguments against it that question whether the influence of the cosmological constant has already been taken into account in the angular diameter distances and impact parameter in the formula \citep{arakida2012effect,butcher2016no,piattella2016lensing}. A large portion of the conflict comes from the fact that coordinate angles are not necessarily physically measurable---indeed, most of the disagreements are about how the mathematical results should translate into observable quantities, especially since the pioneering analyses use a static metric that do not take into account the relative movement of the observer and source. 

% The conflict comes from the fact that coordinate angles are not necessarily physically measurable, and the question of measurable angles was explored by \citet{lebedev2013influence} in detail. 

To date, most of the work done on this topic has been analytical. This inevitably lend some of the work to criticisms of whether the approximations used in deriving the results are valid (see for example criticisms in \citet{ishak2010more}). There have been some numerical work on this subject, but they have been few and far from comprehensive. For example, \citet{beynon2012testing} adopted the use of a Lema{\^\i}tre-Tolman-Bondi metric to model an overdensity in an expanding background and numerically integrated null geodesics in this model, but did not reach a definitive conclusion. More recently \citep{aghili2017effect} used the McVittie metric \citep{mcvittie1933mass} in analyzing the effect of $\Lambda$. 

The most similar work was done by \citet{schucker2009strong}, who adopted a partially numerical approach in the model we are using (the Swiss-cheese model) and concluded he agrees with Rindler and Ishak, but he only uses a single numerical example to reach a conclusion. Furthermore, some higher-order terms were dropped out in the integration in the Kottler metric and the contributiong of $\Lambda$ to the deflection was not singled out. He compares the results of \emph{different} models (pure Kottler versus Swiss-cheese), both involving a non-zero $\Lambda$, to observed quantities, and hence his result are more inclined towards answering the question of which is the right physical model for gravitational lensing, as compared to the question we hope to answer, which is: What is the influence of the cosmological constant given a \emph{single} model, which we assume for the purpose of the analysis, and not without basis, is an accurate model of our universe? On this note we take a slightly different approach from Sch{\"u}cker, where we work with one model (the Swiss-cheese), and compare the results from a $\Lambda = 0$ universe with a universe in which we have a non-zero $\Lambda$. 

Our work will be numerical, and we hope to tackle some of the shortcomings of the previous numerical work, without falling into the approximation traps that exist in analytical work. In addition, we use a Swiss-cheese model that deals with the problem of comoving observers in a cosmological setting and we use observable quantities throughout to compare the effect of $\Lambda$. The details of the model will be explained further in \autoref{chapter:swiss-cheese}.

\section{Structure of this report}
In the next chapter I give an introduction of General Relativity and the basics of light propagation, which lays the mathematical foundation for this work. Following that, I provide an overview of the current established literature on gravitational lensing in a Schwarzschild spacetime where $\Lambda = 0$ and derivation of the key equations, before moving on to the case of a non zero $\Lambda$. 

The bulk of the work is in \autoref{chapter:swiss-cheese}, where I give the mathematical derivation of the equations which form the basis of this project, some of which do not appear explicitly in literature. In particular, I describe the construction and mathematical properties of the Swiss-cheese model with a Kottler condensation, and based on that, obtain the equations for light propagation in such a universe. 

Finally, in \autoref{chapter:results}, I present my numerical results for light propagation in such a universe and discuss their significance in the context of some of the previous analytical analyses. 

% notation
\section{A note on units and notation}
I use a comma to denote partial derivative and an overdot to denote derivative with respect to the affine parameter $\lambda$. For example, $x_{,t} = \frac{\partial x}{\partial t}$ and $\dot{x} = \frac{dx}{d\lambda}$. Throughout this work I use natural units $c = G = 1$. 

%!TEX root = ../thesis.tex
\chapter{Gravitational lensing formalism}
\label{chapter:gravitational-lensing-formalism}

\section{Mathematical preliminaries}

The essence of General Relativity (GR) is very elegantly summarized by John Wheeler \citep[pg.235]{wheeler2000geons} into two parts: matter tells spacetime how to curve, and spacetime tells matter how to move. 

The first half of this statement is quantified by the Einstein Field Equations (EFEs), which is the analogue of Poisson's equation in Newtonian gravity. They are a group of 10 coupled differential equations that describe the interaction between matter and geometry of spacetime, given by (in tensor notation)

\begin{equation}
  G_{\mu \nu} + \Lambda g_{\mu \nu} = 8\pi T_{\mu \nu}
  \label{eq:efes}
\end{equation}
where $g_{\mu \nu}$ is the metric of spacetime, $\Lambda$ is the cosmological constant, $G_{\mu \nu}$ is the Einstein tensor and $T_{\mu \nu}$ is the energy-momentum tensor. The energy momentum tensor on the right hand side is a source term that encodes how matter is distributed in the universe, and on the left hand side the Einstein tensor depends on the metric tensor, which describes the spacetime geometry. For a perfect pressureless fluid, the energy momentum tensor is

\begin{equation}
  T^{\mu \nu} = \rho u^{\mu} u^{\nu}
\end{equation}
where $u^{\mu}$ is the 4-velocity of the fluid. 

If we solve Einstein's field equations, we can obtain the metric tensor $g_{\mu \nu}$, which encodes the spacetime geometry. The metric tensor then influences how a particle behaves in this spacetime, bringing us to the second part of the statement: spacetime tells matter how to move. Given a metric tensor $g_{\mu \nu}$, we can first write down the line element

\begin{equation}
  ds^2 = g_{\mu \nu} dx^{\mu} dx^{\mu}
  \label{eq:line-element}
\end{equation}

Equations of motion of a particle moving in this spacetime can then be derived by first considering the Lagrangian of this particle

\begin{equation}
  \mathcal{L} = \sqrt{g_{\mu \nu} \frac{dx^{\mu}}{d \lambda} \frac{dx^{\nu}}{d \lambda}}
\end{equation}
where $\lambda$ is an affine parameter which increases monotonically along the particle's worldline and $x^{\mu}(\lambda)$ describes the trajectory of the particle. Between two spacetime points $A$ and $B$, we want to maximize $\int^{A}_{B} L(x^{\mu}, \dot{x}^{\mu})\, d \lambda$, so $\mathcal{L}$ satisfies the Euler-Lagrange (E-L) equations

\begin{equation}
  \frac{\partial \mathcal{L}}{\partial x^{\mu}} - \frac{d}{d \lambda}\left ( \frac{\partial \mathcal{L}}{\partial \dot{x}^{\mu}} \right ).
  \label{eq:euler-lagrange-eqn}
\end{equation}

From the E-L equations we arrive at the geodesic equation

\begin{equation}
  \ddot{x}^{\mu} + \Gamma^{\mu}_{\alpha \beta} \dot{x}^{\alpha} \dot{x}^{\beta} = 0 
  \label{eq:geodesic-eqn}
\end{equation}
where an overdot represents a derivative with respect to the affine parameter $\lambda$, and $\Gamma$ are the Christoffel symbols given by

\begin{equation}
  \Gamma^{\mu}_{\alpha \beta} = \frac{1}{2} g^{\mu \rho} (g_{\rho \alpha, \beta} + g_{\rho \beta, \alpha} - g_{\alpha \beta, \rho}).
  \label{eq:christoffels}
\end{equation}

A freely moving particle always move along a geodesic, which is a generalisation of the notion of "straight lines" to a curved spacetime. In addition, light, as a massless particle, travels along null geodesic, in contrast to timelike ones for massive particles. Therefore, all light trajectories have to satisfy the null condition

\begin{equation}
  g_{\mu \nu} dx^{\mu} dx^{\nu} = 0.
  \label{eq:null-condition}
\end{equation}

The result is a handful of coupled differential equations that need to be solved to find the trajectory of light. Nevertheless, a common problem arising in cosmology stems from the fact that as soon as we depart from the simplest homogeneous models used by observational cosmologists, the task of finding solutions to null geodesics quickly becomes an intractable analytical problem. In our work, we integrate the null geodesics numerically to find the trajectory taken by the photon. 

\section{Derivation of bending angle in the Schwarzschild metric}

It is useful to first revise gravitational lensing in a universe without $\Lambda$, in a Schwarzschild metric, which is well understood. The Schwarzschild metric, one of the first known solutions to Einstein's field equations, describes the vacuum that lies outside a spherically symmetric distribution of matter. Its line element is given by

\begin{equation}
  ds^2 = -\left ( 1- \frac{2M}{r} \right ) dt^2 + \left ( 1 - \frac{2M}{r}\right )^{-1} dr^2 + r^2(d\theta^2 + \sin^2\theta d \phi^2).
  \label{eq:schwarzschild-metric}
\end{equation}
where $M$ is the central mass. 

Due to spherical symmetry, we can restrict ourselves to the equatorial plane $\theta = \pi/2$ without loss of generality. This metric is asymptotically flat as $r \rightarrow \infty$. We can then find the total deflection angle $\alpha$ experienced by light that comes in from $r=-\infty$, gets deflected, and travels on towards $r=+\infty$ as 

\begin{equation}
  \alpha = 2 \int_{r_0}^{\infty} \left |  \frac{d\phi}{dr} \right | dr - \pi
\end{equation}
where $r_0$ is the distance of closest approach. 

The static nature and spherical symmetry of the Schwarzschild metric implies that there are two constants of motion for any particle traveling in this geometry. These can be obtained through direct application of the E-L equation (\autoref{eq:euler-lagrange-eqn}), and we have

\begin{subequations}
  \begin{align}
    E &= \left ( 1 - \frac{2M}{r} \right ) \dot{t},\\
    L &= r^2\dot{\phi}.
  \end{align}
  \label{eq:schwarzschild-constants}
\end{subequations}

By applying the null condition (\autoref{eq:null-condition}) on the metric, we obtain an expression for $\frac{d\phi}{dr}$

\begin{equation}
  \frac{d\phi}{dr} = \pm \frac{1}{r^2} \sqrt{\frac{1}{ \frac{1}{b^2} - \left (1- \frac{2M}{r} \right )\frac{1}{r^2} }}
  \label{eq:dphi-dr}
\end{equation}
where $b = L/E$ is the impact parameter (since $\frac{d\phi}{dr} = \dot{\phi}/\dot{r}$). Integrating this (for a detailed derivation see \cite{keeton2005formalism}), we obtain an expression for the bending angle $\alpha$ as a series expansion in $M/r_0$. This series, to third order in $M/r_0$, is as follows:

\begin{equation}
  \alpha = 4 \frac{M}{r_0} + \left ( -4 + \frac{15\pi}{4} \right )\left ( \frac{M}{r_0}\right )^2 + \left ( \frac{122}{3} - \frac{15\pi}{2} \right )\left ( \frac{M}{b}\right )^2.
  \label{eq:lensing-series-expansion-r0}
\end{equation}

This can be easily converted to a series in $M/b$ instead of $M/r_0$, which is usually done in literature, since a relation beteen $b$ and $r_0$ can be derived in a series expansion by setting $\dot{r} = 0$, giving \citep{keeton2005formalism}

\begin{equation}
  r_0 = b \left [ 1 - \frac{M}{b} - \frac{3}{2} \left ( \frac{M}{b}\right)^2 - 4\left ( \frac{M}{b}\right)^3 \right ].
  \label{eq:b-r0-relation}
\end{equation} 

This relation can then be used to rewrite the expansion in terms of $\frac{M}{b}$ to give us

\begin{equation}
  \alpha = 4 \frac{M}{b} + \frac{15\pi}{4} \left ( \frac{M}{b} \right )^2 + \frac{128}{3} \left ( \frac{M}{b} \right )^3.
  \label{eq:series-expansion-b}
\end{equation}
and indeed many authors use the impact parameter to discuss the bending of light in Schwarzschild spacetime \citep{wald2010general,misner2017gravitation,butcher2016no}. However, it has been pointed out in literature \citep{ishak2008new,hammad2013note,lebedev2013influence} that in the non-zero $\Lambda$ case, the definition of the impact parameter is no longer independent of $\Lambda$ and becomes questionable due to the fact that spacetime is no longer asymptotically flat. 

In this work we will work with another constant of the motion for the non-zero $\Lambda$ case, and for consistency we will use that parameter here as well so that a fair comparison can be made, even though the Schwarzschild case does not include a cosmological constant.

We can define another constant of the motion $R_u$ which corresponds to the unperturbed trajectory of light (see \autoref{fig:lensing}), which is related to $r_0$ by (see Eq. 6 of \citet{ishak2008new} and Eq. 3 of \citet{butcher2016no}).

\begin{equation}
  \frac{1}{r_0} = \frac{1}{R_u} + \frac{M}{R_u^2} + \frac{3M^2}{16R_u^3}
  \label{eq:r0-R-relation}
\end{equation}
where we have added a subscript $u$ to differentiate it from the Kottler coordinate $R$ in the next chapter. Using this relation and applying it to \autoref{eq:lensing-series-expansion-r0}, we can get a series expansion in $M/R_u$, which, up to third order, is given by

\begin{equation}
  \alpha = 4 \frac{M}{R_u} + \frac{15\pi}{4} \left ( \frac{M}{R_u} \right )^2 + \frac{401}{12} \left ( \frac{M}{R_u} \right )^3.
  \label{eq:series-expansion-R}
\end{equation}

There are 3 length quantities that are typically used in Schwarzschild lensing: the distance of closest approach $r_0$, the impact parameter $b$, and the parameter of the unperturbed trajectory $R_u$. They are related to each other through \autoref{eq:b-r0-relation} and \autoref{eq:r0-R-relation}, and their related series expansions are respectively given by \autoref{eq:lensing-series-expansion-r0}, \autoref{eq:series-expansion-b} and \autoref{eq:series-expansion-R}. The coefficient of the leading order term is the same for all three but they differ on higher order terms in the series expansion. Many gravitational lensing analysis done on the Schwarzschild metric are only concerned with the leading order term, and hence use these three lengths somewhat interchangeably. But in this project we are interested in corrections at the second order or higher, and the distinction becomes important. In the subsequent section of the report, we will be using the series expansion in $M/R_u$ (\autoref{eq:series-expansion-R}). 

As mentioned previously, \citet{rindler2007contribution} proposed a different expression for $\alpha$. The expression was later refined for a Swiss-Cheese universe \citep{ishak2008new}, and I will state it here (in our notation) for easy comparison:

\begin{equation}
  \alpha_{\text{Ishak}} = 4 \frac{M}{R_u} + \frac{15\pi}{4} \left ( \frac{M}{R_u} \right )^2 + \frac{305}{12} \left ( \frac{M}{R_u} \right )^3 - \frac{\Lambda R R_h}{3},
  \label{eq:rindler-ishak-alpha}
\end{equation}
where $R_h$ is the boundary of the hole in static coordinates in the Swiss-Cheese model. The $\Lambda$-term is negative, and they postulate that the cosmological constant attenuates lensing. 

\section{Lensing observables}

A significant part of the conflict in literature comes from the question of which quantities in lensing are observable, and whether these observable quantities are ultimately affected by the presence of the cosmological constant. Therefore we aim to stick with strictly observable quantities. 

\begin{figure}
  \centering
  \includegraphics[height=0.4\linewidth]{images/lensing_cropped.pdf}
  \caption{Diagram of gravitational lensing, where the lens, observer, and source are collinear.}
  \label{fig:lensing}
\end{figure}

We consider a simple picture where the observer, lens, and source are aligned, as shown in \autoref{fig:lensing}. Bending is assumed to happen at a single point above the lens, since the distance from the observer to the lens and source is assumed to be much larger than the distance of closest approach between the light ray and the lens. From this diagram we can easily obtain the lens equation \citep{schneider1992gravitationallenses} with some trigonometry

\begin{equation}
  D_S \theta_E = D_{LS} \alpha
  \label{eq:lens-eqn}
\end{equation}
where $\alpha$ is the bending angle as previously derived, $D_S$ is the angular diameter distance from observer to source, $D_{LS}$ is the angular diameter distance from lens to source, and $\theta_E$ is known as the Einstein angle. $R_u$ can also be expressed in terms of the observable quantities $R_u = D_L \theta_E$, where $D_L$ is the angular diameter distance from observer to the lens.  

With this in mind, we can begin looking at the Swiss-Cheese model, and look to apply this formalism in such a universe. 
%!TEX root = ../thesis.tex
\chapter{Description of the Swiss Cheese model}
\label{chapter:swiss-cheese}

\section{Spacetime patches}

Swiss-Cheese (SC) models model were first introduced by \citet{einstein1945influence} to investigate the gravitational field of a mass well described by the Schwarzschild metric but embedded in a non-Minkowski background spacetime. Such a model is contructed by removing a coming sphere from the homogeneous and replacing it with an inhomogeneous mass distribution. In this case, we use a mass distribution that is vacuum everywhere except for a point mass at the centre. In principle, since the sphere is comoving, multiple spheres can be inserted in the cheese as long as they are initially non-overlapping. In our model, only one hole is needed to model the lens. This stitching of two metrics on the `cheese'-`hole' boundary is of course not arbitrary, and matching conditions will impose restriction on the parameters of the two metrics. This will be discussed in detail in the following sections. 

There are several reasons why this model was chosen. First, this is an exact solution of Einstein's equations which preserves the global dynamics, and hence it will allow us to properly investigate the higher order corrections that have been oft-debated in literature. Previously some research \citep{simpson2010lensing}[??] has been done using a perturbative approach, but others \citep{ishak2010more} have contested whether the approximations were valid. Secondly, by putting observers in the homogeneously expanding "cheese", it accounts for observers moving with the Hubble flow, which is a common objection to Rindler and Ishak's use of a static metric \citep{simpson2010lensing,butcher2016no,park2008rigorous,khriplovich2008does}. Lastly, this model also takes the finite range of the mass into account by confining the influence of the central mass to the size of the hole. 

Light propagation in SC models has been extensively studied \citep{szybka2011light,vanderveld2008luminosity,fleury2014swiss}, but not particularly so in the subject of Lambda's dependence on gravitational lensing. Some of the areas that Swiss-Cheese models have been commonly used include investigating the effect of local inhomogeneities on luminosity-redshift relations \citep{kantowski1969corrections,fleury2013interpretation} and studying fluctuations in redshift and distance of the cosmic microwave background \citep{bolejko2009szekeres,valkenburg2009swiss,bolejko2011effect}.


% In the Swiss-cheese model, in principle multiple holes can be inserted into the cheese. In this project, we use a single hole for the lens. 

\subsection{Friedmann-Robertson-Walker geometry}

Outside the hole, geometry is described by the Friedmann-Robertson-Walker (FRW) metric, the simplest homogeneous and isotropic model of the universe. Its line element is given by

\begin{equation}
  ds^2 = -dt^2 + a(t)^2 \left ( \frac{dr^2}{1-kr^2} + r^2 d \Omega^2 \right )
  \label{eq:frw-metric}
\end{equation}
where $d \Omega^2 = d\theta^2 + \sin^2\theta d\phi^2$ is the metric on a 2-sphere, $a$ is the scale factor, and $k$ represents the curvature. This is the line element for a homogeneous and expanding universe, with general spatial curvature. The scale factor $a$ parametrizes the relative expansion of the universe, such that the relationship between physical distance and comoving distance between two points at a certain cosmic time $t$ is given as

\begin{equation}
  d_{\text{physical}} = a(t) d_{\text{comoving}}.
  \label{eq:comoving-physical-distance}
\end{equation}

The scale factor also satisfies the Friedmann equation

\begin{equation}
  H^2 \equiv \left ( \frac{a_{,t}}{a} \right ) = \frac{8\pi G \rho}{3} + \frac{\Lambda}{3} - \frac{k}{a^2}
  \label{eq:friedmann-equation}
\end{equation}
where $\rho$ is the energy density of a pressureless fluid and $H$ is the Hubble parameter. 

It is common to introduce the cosmological parameters, where a subscript 0 refers to quantities evaluated today: 

\begin{equation}
  \Omega_m = \frac{8\pi G \rho_0}{3H_0^2}, \,\, \Omega_{\Lambda} = \frac{\Lambda}{3H_0^2}, \,\, \Omega_k = - \frac{k}{a_0^2 H_0^2}
  \label{eq:cosmo-params}
\end{equation}

and rewrite the Friedmann equation as

\begin{equation}
  H^2 = H_0^2 \left [ \Omega_m \left ( \frac{a_0}{a}\right )^3 + \Omega_k \left ( \frac{a_0}{a}\right )^2 + \Omega_{\Lambda} \right ]. 
  \label{eq:friedmann-eqn-version2}
\end{equation}

Subsequently in this report instead of working directly with $\Lambda$ I will work with $\Omega_{\Lambda}$ instead. 

\subsection{Kottler geometry}

In this project we use a Kottler condensation in the Swiss-Cheese for the central lensing mass. This is described by a Kottler metric \citep{kottler1918physikalischen}, which is the extension of the famous Schwarzschild metric to include a cosmological constant, given by

\begin{equation}
  ds^2 = -f(R)dT^2 + \frac{dR^2}{f(R)} + R^2 d \Omega^2
  \label{eq:kottler-metric}
\end{equation}
with
\begin{equation}
  f(R) = 1-\frac{2M}{R} - \frac{\Lambda R^2}{3},
  \label{eq:kottler-metric-f}
\end{equation}
where M is the mass of the central object. Unlike the FRW, this metric describes a static spacetime. 

\section{Matching conditions}

Two geometries can be matched across the boundary to form a well defined spacetime only if and only if they satisfy the Darmois-Israel junction conditions \citep{darmois1927equations,israel1966singular}. These conditions dictate that the first and second fundamental forms of the two metrics must match on the matching hypersurface $\Sigma$, that is, both metrics must induce 
\begin{inparaenum}[(i)]
  \item the same metric, and 
  \item the same extrinsic curvature.
\end{inparaenum}

\subsection{Continuity of the induced metric}

We match the FRW and Kottler metrics on a surface of a comoving 2-sphere, $\Sigma$, which is defined by $r = r_h = \text{constant}$ in FRW coordinates and $R = R_h(T)$ in Kottler coordinates.

The induced metric is the quantity 

\begin{equation}
  h_{ab} = g_{\alpha \beta} j^{\alpha}_{a} j^{\alpha}_{\beta}
  \label{eq:induced-metric-defn}
\end{equation}
where $j^{\alpha}_{a}$ is defined as

\begin{equation}
  j^{\alpha}_{a} = \frac{\partial \bar{X}^{\alpha}}{\partial \sigma^a}.
  \label{eq:j-defn}
\end{equation}
Here we have introduced $X^{\alpha}$ to represents coordinates of the original metric. We define $\sigma^a$ to be natural intrinsic coordinates for $\Sigma$, and $\bar{X}^{\alpha}(\sigma^a)$ is the parametric equation of the hypersurface. 

More concretely, using the coordinates defined previously in \autoref{eq:frw-metric}, these quantities are

\begin{subequations}
  \begin{align}
    X^{\alpha} &= \{ t, r, \theta, \phi \} \\
    \sigma^a &= \{ t, \theta, \phi \} \\
    \bar{X}^{\alpha}(\sigma^a) &= \{ t, r_h, \theta, \phi\}.
  \end{align}
\end{subequations}

Similarly, in the Kottler region, we have 

\begin{subequations}
  \begin{align}
    X^{\alpha} &= \{ T, R, \theta, \phi \} \\
    \sigma^a &= \{ T, \theta, \phi \} \\
    \bar{X}^{\alpha}(\sigma^a) &= \{ T, R_h(T), \theta, \phi\}.
  \end{align}
\end{subequations}

Using these definitions, the 3-metric induced by the FRW geometry on $\Sigma$ is

\begin{equation}
  ds^2_{\Sigma} = -dt^2 + a^2(t)r^2 d \Omega^2,
  \label{eq:frw-induced-metric}
\end{equation}
while the induced metric on the Kottler metric is
\begin{equation}
  ds_{\Sigma}^2 = -\kappa^2(T)dT^2 + R_h^2(T) d \Omega^2,
  \label{eq:kottler-induced-metric}
\end{equation}
where
\begin{equation}
  \kappa \equiv \sqrt{\frac{f^2[R_h(T)] - R_{h,T}^2(T)}{f[R_h(T)]}}.
  \label{eq:kottler-kappa}
\end{equation}

Equating the components of \autoref{eq:frw-induced-metric} and \autoref{eq:kottler-induced-metric}, we obtain the following:

\begin{equation}
  R_h(T) = a(t)r
  \label{eq:r-to-ar},
\end{equation}

\begin{equation}
  \frac{dt}{dT} = \kappa(T).
  \label{eq:dt-dT}
\end{equation}

% \begin{subequations}
%   \begin{align}
%     R_h(T) = a(t)r \\
%     \frac{dt}{dT} = \kappa(T).
%   \end{align}
% \end{subequations}

These two relationships relate the radial and time coordinates of the two metrics respectively. 

\subsection{Continuity of the extrinsic curvature}

The second condition equates extrinsic curvature of the two geometries. By definition, the extrinsic curvature $K_{ab}$ of a hypersurface is given by

\begin{equation}
  K_{ab} = n_{\alpha;\beta} j^{\alpha}_{a} j^{\beta}_{a}
  \label{eq:extrinsic-curvature-defn}
\end{equation}
where $n_{\mu}$ is the unit vector normal to $\Sigma$, $j$ is as defined previously in \autoref{eq:j-defn},and the semicolon notation ``;'' denotes a covariant derivative, for example, $n_{\alpha;\beta} = \nabla_{\beta}\, n_{\alpha}$. For any vector $V^{\nu}$, the covariant derivative is defined as

\begin{equation}
  \nabla_{\mu} = \partial_{\mu}V^{\nu} + \Gamma^{\nu}_{\mu \rho} V^{\rho}.
  \label{eq:covariant-derivative-defn}
\end{equation}

For a hypersurface defined by a function $q = 0$, the unit vector normal to it is

\begin{equation}
  n_{\mu} = \frac{q_{,\mu}}{\sqrt{g^{\alpha \beta} f_{,\alpha} f_{,\beta}}}.
  \label{eq:unit-normal-vector}
\end{equation}

In our case $q = r-r_h$ in FRW coordinates and $q = R - R_h(T)$ in Kottler coordinates. For example, the unit vector in the FRW region is trivial to calculate, and we get $n_{\mu}^{\text{(FRW)}} = \delta^r_{\mu}/a$. Applying this formula, the extrinsic curvature induced by the FRW geometry is

\begin{equation}
  K_{ab} dx^a dx^b = \frac{a(t)r}{\sqrt{1-kr^2}} d \Omega^2
  \label{eq:extrinsic-curvature-frw}
\end{equation}
while the extrinsic curvature induced by the Kottler geometry is
\begin{equation}
  K_{ab} dx^a dx^b = \frac{1}{\kappa} \left [ R_{h,tt} + \frac{f^{\prime}}{2f}(f^2 - 3R_{h,t}^2) \right ] dT^2 + \frac{R_h f}{\kappa} d \Omega^2
  \label{eq:extrinsic-curvature-kottler}
\end{equation}
where $f^{\prime} = \partial f / \partial R$, and all quantities are evaluated at $R = R_h(T)$.

Equating the components of \autoref{eq:extrinsic-curvature-frw} and \autoref{eq:extrinsic-curvature-kottler}, we obtain

\begin{equation}
  \frac{R_h f}{\kappa} = \frac{a(t)r}{\sqrt{1-kr^2}}
  \label{eq:kappa-to-fprime}
\end{equation}
and
\begin{equation}
  R_{h,tt} + \frac{f^{\prime}}{2f}(f^2 - 3R_{h,t}^2) = 0.
\end{equation}
The second equation is provided for completeness although it is not needed for subsequent derivations. 

\subsection{Consequences on the property of the hole}

Combining \autoref{eq:kappa-to-fprime} with \autoref{eq:kottler-kappa}, we can eliminate $\kappa$. We can also replace $R_{h,T}$ using the relation obtained in \autoref{eq:r-to-ar}, since

\begin{equation}
  \frac{dR_h}{dT} = \frac{d(ar)}{dT} = \frac{da}{dt}\frac{dt}{dT}r.
\end{equation}
where $da/dt$ is given by the Friedmann equation \ref{eq:friedmann-equation}. 

Following through the algebra, we arrive at the somewhat intuitive result that both regions must have the same cosmological constant $\Lambda$ and that the central mass $M$ in the Kottler region must be equal to the original mass inside the homogeneous comoving sphere of radius $r_h$

\begin{equation}
  M = \frac{4\pi}{3} a^3 r_h^3. 
  \label{eq:junction-conditions-mass-volume}
\end{equation}

The last thing we need from the boundary conditions is the relate the tangent vectors between the two metrics. The continuity of the metric, imposed by the first junction condition, implies that the connection does not diverge across the boundary. Therefore, light is not deflected as it crosses the boundary and we just need to convert the components of the tangent vector between the two coordinate systems. To obtain $\dot{R}$ in terms of FRW tangent vectors $\dot{r}$ and $\dot{t}$, we differentiate \autoref{eq:r-to-ar} and substitute $a_t$ with the Friedmann equation \autoref{eq:friedmann-equation}. Keeping in mind the boundary conditions, we get an expression for $\dot{R}$. The angular coordinates and angular tangent vectors are unchanged moving from Kottler to FRW coordinates, and vice versa. With $\dot{R}$ and $\dot{\phi}$, $\dot{T}$ then can be easily obtained from the null condition \autoref{eq:null-condition}. The result is

\begin{subequations}
  \begin{align}
    \dot{T} &= \frac{1}{f}\sqrt{1-kr^2} \dot{t} + \frac{a}{f\sqrt{1-kr^2}} \sqrt{\frac{2M}{ar} - kr^2 + \frac{\Lambda}{3}a^2 r^2} \dot{r} \\
    \dot{R} &= \sqrt{\frac{2M}{ar} - kr^2 + \frac{\Lambda}{3}a^2 r^2} \dot{t} + a\dot{r}\\
    \dot{\phi} &= \dot{\phi}\\
    \dot{\theta} &= \dot{\theta}
  \end{align}
  \label{eq:kottler-to-frw-transform}
\end{subequations}
where for completeness I have also given the trivial relations between the angular tangent vectors. The quantities above are all evaluated at the boundary of the hole. This result is given for flat space in \citet{fleury2013interpretation} and \citet{schucker2009strong}, but here it has been extended to allow for arbitrary spatial curvature. The reverse transformation is easily obtained by inverting the Jacobian from above. 

In summary, given a FRW spacetime with pressureless matter and a cosmological constant $\Lambda$, a spherical hole, whose geometry is described by the Kottler metric, can be constructed which contains a constant mass $M = 4\pi \rho a^3 r_h^3/3$ at its centre. The geometry resulting from combining the two metrics at the boundary is an exact solution of the Einstein field equations. Applying the boundary conditions, we can obtain all the necessary transformations needed for continuation of light propagation at the boundary. 

\section{Light propagation}

Light propagation is governed by the geodesic equation. 

Due to spherical symmetry, we can restrict ourselves to the $\theta = \pi/2$ plane without loss of generality. 
%!TEX root = ../thesis.tex
\chapter{Results and discussion}
\label{chapter:results}

We were able to reproduce the results presents in \citet{schucker2009strong}. 

A graph of results when we keep the lensing mass $M$ constant and vary $\Omega_{\Lambda}$ can be seen in [fig??]. On the $y$-axis, we have plotted the the deviation of $\alpha$ as a fraction of the standard FRW lensing case (given by \autoref{eq:series-expansion-R}), in order to put them on the same scale. [State the mass and zlens parameters] 

By varying the integration step size, we are able to estimate numerical errors on the integration. For a certain step size, we group the results obtained from the vicinity of step sizes together and find the variance in bending angles in that range. Fig [??] shows how the $\alpha$ obtained varies with step size. As is expected, the precision increases as we reduce the step size. 

Our results seem to follow the trend of Kantowski's predictions most closely, with a gap that reduces towards higher $\Lambda$. A possible explanation of this gap can be found by examining the neglected higher order term in Kantowski's predicted bending angle [eq??]. When $\Lambda = 0$, the ratio of this term to the leading order $(4M/r_0^2) \cos^3 \tilde{\phi_1}$ term is of the same order of magnitude as the fractional deviation of our numerical results from Kantowski's predictions. As is expected, this ratio decreases as $\Lambda$ when mass is kept constant, as can be seen from [fig??]. 

From the graph, we can see that even for the $\Lambda = 0$ case there is an offset between the numerical Swiss-Cheese result and the FRW prediction. Qualitatively, this is due to the fact that conventional lensing analyis assumes a mass superimposed on the homogeneous background, and this mass has infinite range. However, in the Swiss-Cheese model, the influence of the mass is limited, and bending stops once it leaves the Kottler hole. This is the main effect that Kantowski quantified in his paper \citet{kantowski2010gravitational}. This then begs the question of which model is a more accurate description of our physical universe, but this is not our primary concern. We are concerned about whether $\Lambda$ has an influence on this effect. 

There are a few different factors at play here. In discussing the results of this numerical integration, let us take a step back to look at the specific parts of ray-tracing that have a $\Lambda$-dependence. These are:
\begin{enumerate}
  \item The size of the hole. This is governed by \autoref{eq:junction-conditions-mass-volume}. In flat space, increasing $\Omega_{\Lambda}$ implies decreasing $\Omega_{\Lambda}$, which corresponds to the matter density of the universe. If we are to keep the mass constant, the hole size would have to increase as we increase $\Omega_{\Lambda}$. 
  \item The rate of expansion of the hole in static Kottler coordinates, given by \autoref{eq:hole-expansion-in-kottler-dR-dT}.  
  \item The Jacobian at the boundary, given by \autoref{eq:kottler-to-frw-transform-jacobian}.
\end{enumerate}

The first effect does not seem to be a truly direct $\Lambda$ effect, merely a side effect that in a flat universe, changing $\Omega_{\Lambda}$ must imply a change in matter density, but ultimately, it is the size of the hole that is the true determining factor. 



\newpage
\noindent
\vspace*{2cm}
\bookmarksetup{startatroot}
\printbibliography[heading=bibintoc, title={References}]

\end{document}
